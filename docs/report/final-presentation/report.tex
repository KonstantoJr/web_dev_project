\documentclass{acmart}   
\usepackage[utf8]{inputenc} 
\usepackage[greek , english]{babel}
\usepackage{alphabeta}
\usepackage{listings}
\usepackage{color}
\usepackage{float}
\graphicspath{{./img/}}
\definecolor{mygreen}{rgb}{0,0.6,0}
\definecolor{mygray}{rgb}{0.5,0.5,0.5}
\definecolor{mymauve}{rgb}{0.58,0,0.82}

\begin{document}
       \begin{titlepage}
              \begin{center}
              \vspace*{1cm}

              \line(1,0){\textwidth}\\
              \textbf{Πρόξεκτ Ομάδας 22}\\
              \vspace{0.5cm}
              ΙΣΤΟΣΕΛΊΔΑ ΚΡΑΤΉΣΕΩΝ
              \vspace{1.5cm}
              \line(1,0){\textwidth}\\
              \textbf{Κόντος Παναγιώτης}\\
              Τμήμα: Ηλεκτρολόγων Μηχανικών \& Τεχνολογίας Υπολογιστών\\
              Τομέας: Υπολογιστές\\
              ΑΜ: 1066620\\
              Έτος Φοίτησης: 4ο\\
              \vspace{0.8cm}
              \textbf{Κωνσταντίνος Κωνσταντόπουλος}\\
                     Τμήμα: Ηλεκτρολόγων Μηχανικών \& Τεχνολογίας Υπολογιστών\\
              Τομέας: Υπολογιστές\\
              ΑΜ: 1066546\\
              Έτος Φοίτησης: 4ο\\

              \vspace{0.8cm}

              % \includegraphics[width=0.4\textwidth]{university}

              
              \end{center}
       \end{titlepage}
\tableofcontents

\section{Εισαγωγή}
Το πρότζεκτ που μας ανατέθηκε ήταν να φτιάξουμε μια εφαρμογή για τις υπηρεσίες που παρέχονται στους φοιτητές του Πανεπιστημίου Πατρών. Οι υπηρεσίες αυτές όπως φαίνεται και στην ιστοσελίδα της φοιτητικής μέριμνας http://dfm.upatras.gr/  είναι η σίτιση και η στέγαση και παρέχονται δωρεάν από το κράτος στους φοιτητές που πληρούν τα απαιτούμενα κριτήρια.  Το κύριο κριτήριο είναι το οικογενειακό εισόδημα και γι’ αυτό κάθε χρόνο γίνεται επανεξέταση των δικαιούχων. Το πιο σημαντικό είναι να γίνεται πιο εύκολα ο έλεγχος των αιτήσεων ώστε να χρειάζεται λιγότερος χρόνος για την εκτίμηση τους.
\section{Μεθοδολογια}
\subsection{Στόχος}
\end{document}