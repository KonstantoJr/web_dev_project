\documentclass{acmart}   
\usepackage[utf8]{inputenc} 
\usepackage[greek , english]{babel}
\usepackage{alphabeta}
\usepackage{hyperref} 
\usepackage{listings}
\usepackage{color}
\usepackage{float}
\graphicspath{{./img/}}
\definecolor{mygreen}{rgb}{0,0.6,0}
\definecolor{mygray}{rgb}{0.5,0.5,0.5}
\definecolor{mymauve}{rgb}{0.58,0,0.82}

\begin{document}
       \begin{titlepage}
              \begin{center}
              \vspace*{1cm}

              \line(1,0){\textwidth}\\
              \textbf{Πρότζεκτ Ομάδας 22}\\
              \vspace{0.5cm}
              ΙΣΤΟΣΕΛΊΔΑ ΚΡΑΤΉΣΕΩΝ
              \vspace{1.5cm}
              \line(1,0){\textwidth}\\
              \textbf{Κόντος Παναγιώτης}\\
              Τμήμα: Ηλεκτρολόγων Μηχανικών \& Τεχνολογίας Υπολογιστών\\
              Τομέας: Υπολογιστές\\
              ΑΜ: 1066620\\
              Έτος Φοίτησης: 4ο\\
              \vspace{0.8cm}
              \textbf{Κωνσταντίνος Κωνσταντόπουλος}\\
                     Τμήμα: Ηλεκτρολόγων Μηχανικών \& Τεχνολογίας Υπολογιστών\\
              Τομέας: Υπολογιστές\\
              ΑΜ: 1066546\\
              Έτος Φοίτησης: 4ο\\

              \vspace{0.8cm}

              % \includegraphics[width=0.4\textwidth]{university}

              
              \end{center}
       \end{titlepage}
\tableofcontents
\newpage
\section{Εισαγωγή}
Το πρότζεκτ που μας ανατέθηκε ήταν να φτιάξουμε μια ιστοσελίδα για κρατήσεις εκδηλώσεων. Οι εκδηλώσεις αυτές φιλοξενούνται απο πολλους φορείς. Στην δική μας υλοποίηση ο κάθε φορέας θα διαθέτει εναν λογαριασμό διαχειριστή τον οποίο εισάγουμε εμείς χειροκίνητα στο σύστημα. Ο κάθε φορέας μπορεί να δει όλες τις εκδηλώσεις αλλα μπορεί να επεξεργαστεί μόνο τις δικές του. Οι χρήστες της ιστοσελίδας βλέπουν τις παραστάσεις και μπορούν να κάνουν την κράτηση που τους ενδιαφέρει.  

\section{Μεθοδολογια}
\subsection{Στόχος}
Ο κύριος στόχος μας ήταν η δημιουργία μιας ιστοσελίδας στην οποια θα υπάρχουν συγκεντρωμένες όλες οι εκδηλώσεις ώστε να μπορεί όποιος επιθυμεί να κλείσει εύκολα και γρήγορα θέση για την παράσταση της αρεσκείας του. Επιπροσθέτως η διαχείριση των εκδηλώσεων απο τους διαχειριστές θα έπρεπε να είναι εύκολη και να μην απαιτούνται ειδικές γνώσεις για την χρήση της ιστοσελίδας.

\subsection{Υλοποίηση}
\subsubsection{Προσέγγιση}\hfill\\
Όταν μας ανατέθηκε το θέμα της εργασίας κανονίσαμε μία συνάντηση με τον κ. Σιντόρη ώστε να μας καθοδηγήσει σχετικά με τις δυνατότητες που θα παρέχει η ιστοσελίδα μας. Αφου επισκεφθήκαμε και παρόμοιες ιστοσελίδες για να δούμε τον τρόπο λειτουργίας τους ξεκινήσαμε να εργαζόμαστε. Επειδή θα ήταν πιο δύσκολη η διεκπεραίωση της εργασίας αν δουλεύαμε στο ίδιο κομμάτι, ο ένας ασχολήθηκε πιο πολυ με το front-end και ο αλλος με το back-end.

\subsubsection{Χρονοδιάγραμμα}\hfill\\
Αρχικά ασχοληθήκαμε με το διάγραμμα οντοτήτων-συσχετίσεων που μας ήταν γνωστό απο το μάθημα των Βάσεων Δεδομένων. Έπειτα ξεκινήσαμε με το να σχεδιάσουμε στο χαρτί ένα προσχέδιο της ιστοσελίδας μας. Με την πάροδο των μαθημάτων και την απόκτηση των απαιτούμενων γνώσεων κατασκευάζαμε βήμα βήμα το front-end. Την τελευταία εβδομάδα που μάθαμε και για τον προγραμματισμό της ιστοσελίδας ενώσαμε τα μέρη της ιστοσελίδας μας, ωστε να είναι λειτουργική.

\subsubsection{Δημιουργια erd και βασης δεδομένων}\hfill\\
Δημιουργήσαμε το διάγραμμα οντοτήτων - συσχετίσεων για τον μικροκοσμο που μας απασχολούσε όπως φαίνεται στην παρακάτω εικόνα.

Επίσης δημιουργήσαμε την βάση δεδομένων όπου θα αποθηκεύονται και θα αντλούνται οι πληροφορίες που χρειαζόμαστε. Στην βάση δέν βάλαμε πολλές εκδηλώσεις, καθώς αυτές πρέπει να εισάγωνται απο τον αντίστοιχο φορέα οργάνωσης.

\subsubsection{Front-end}\hfill\\

\subsubsubsection{Δημιουργια της Αρχικής σελίδας}\hfill\\
Το πρώτο κομμάτι που φτιάξαμε μετά την βάση δεδομένων ήταν η αρχίκη σελίδα. Έχοντας πάρει ιδέες απο παρόμοιες ιστοσελίδες (π.χ \href{www.viva.gr}{Viva}) αποφασίσαμε οτι όλες οι σελίδες της ιστοσελίδας μας θα χρησιμοποιούν ένα λογότυπο με εναν τίτλο και ενα navigation bar με παραπομπές στην αρχική σελίδα (Αρχική), στις εκδηλωσεις (Εκδηλώσεις), στις πληροφορίες της σελίδας (Πληροφορίες), στον πίνακα ελέγχου (Πίνακας Ελέγχου) το οποίο φαίνεται μόνο αν ο συνδεδεμένος χρήστης είναι διαχειριστής, και στην είσοδο (Είσοδος) απο όπου μπορεί να πραγματοποιήσει είσοδο ή εγγραφή στην σελίδα. Επίσης στην αρχική σελίδα φαίνονται οι τρεις δημοφιλέστερες παραστάσεις σε carousel στο κέντρο της σελίδας. Δίνεται η δυνατότητα πατώντας πάνω στην εικόνα να κατευθυνθεί ο χρήστης στην συγκεκριμένη εκδήλωση.


\subsubsubsection{Δημιουργια της όλων των εκδηλώσεων}\hfill\\
Στην συγκεκριμένη σελίδα φαίνονται σε μορφή λίστας όλες οι διαθέσιμες εκδηλώσεις. Υπαρχει μια μικρη φωτογραφία στα αριστερά, ο τίτλος, η ημερομηνία διεξαγωγής και μια μικρή περιγραφή 150 λέξεων. Πατώντας πάνω στην φωτογραφία, στον τίτλο ή στην ημερομηνία ο χρήστης θα μεταφερθεί στην σελίδα της εκδηλωσης.

\subsubsubsection{Δημιουργια σελίδας εκδήλωσης}\hfill\\
Ανοίγοντας την σελίδα της εκδήλωσης στο πάνω μέρος υπάρχει το banner της εκδήλωσης. Ακολουθούν ακριβως απο κάτω ο τίτλος, η ημερομηνία διεξαγωγής, το μέρος προβολής και ένα κουμπί για να κάνει κράτηση ο ενδιαφερόμενος χρήστης. Επιπλέον στο κάτω μέρος υπάρχει ένα μενού από το οποίο ο χρήστης μπορεί να επιλέξει να δει ολόκληρη την περιγραφή της παράστασης, τους συντελεστές, τον διοργανωτή και καποιες άλλες πληροφορίες όπως την τιμή, τις συνολικές και τις διαθέσιμες θέσεις και το τηλέφωνο επικοινωνίας.

\subsubsubsection{Δημιουργια της φόρμας κράτησης}\hfill\\
Εφόσον ο χρήστης επέλεξε να κάνει κράτηση πατώντας το αντίστοιχο κουμπί στην σελίδα της εκδήλωσης, παραμπέμπεται στην φόρμα κράτησης. Τα στοιχεία που θα του ζητηθούν είναι το όνομα του, ενα email και ενα τηλέφωνο επικοινωνίας. Κατόπιν επιλέγει το πλήθος και τον τύπο των εισητηρίων που επιθυμεί και στα δεξιά θα του εμφανιστεί το συνολικό κόστος των εισητηρίων. Για να καταχωρηθεί η κράτηση του θα πρέπει να πατήσει το κουμπί "Ολοκλήρωση Κράτησης". Σε περίπτωση που δεν θέλει να συνεχίσει με την κράτηση, επιλέγοντας ακύρωση γυρνάει στην σελίδα της εκδήλωσης.

\subsubsubsection{Δημιουργια της σελίδας Πληροφορίες}\hfill\\
Στην συγκεκριμήνη σελίδα υπάρχουν πληροφορίες σχετικά με την λειτουργεία της ιστοσελίδας και τους δημιουργους της. Υπαρχει ενα μενού όπου επιλέγοντας "Διαχειρηστής" ή "Χρήστης" φαίνονται οι δυνατότητες του διαχειρηστή και του απλού χρήστη αντίστοιχα.

\subsubsubsection{Δημιουργια της σελίδας του Πίνακα Ελέγχου}\hfill\\
Εφόσον ο χρήστης που θα συνδεθεί είναι διαχειριστής, θα του εμφανιστεί ο πίνακας ελέγχου. Εκεί μπορεί να δεί όσες εκδηλώσεις έχει προσθέσει ο ίδιος παλαιότερα. Στις συγκεκριμένες εκδηλώσεις έχει τρεις επιλογές.

\begin{itemize}
  \item Επεξεργασία, όπου πατώντας το κουμπί του ανοίγει η φόρμα της εκδήλωσης συμπληρωμένη με τα τελευταία στοιχεια που είχε εκχωρήσει. Μπορεί να πραγματοποιήσει τις αλλαγές που επιθυμεί και πατώντας "Αποθήκευση Παράστασης" ή να επιλέξει ακύρωση και θα ανακατευθύνθει πίσω στον πίνακα ελέγχου και μπορεί να δεί τις αλλαγες εφόσον τις πραγματοποίησε. 
  \item Διαγραφή, όπου σβήνει απο την βάση την συγκεκριμένη εκδήλωση.
  \item Κρατήσεις, όπου θα του ανοίξει μια νέα σελίδα και θα δει σε έναν πίνακα τις υπάρχουσες κρατήσεις και τα στοιχεία τους.
\end{itemize}

Επιπροσθέτως μπορεί να κάνει δήμιουργία μιας νέας εκδήλωσης πατώντας στο κουμπί που βρίσκεται στο πάνω μέρος της οθόνης.

\subsubsubsection{Δημιουργια της φόρμας Εκδήλωσης}\hfill\\
Αφού ο διαχειριστής επέλεξε να δημιουργήσει μια νέα εκδήλωση, του εμφανίζεται η φόρμα της εκδήλωσης. Εκεί πρέπει να εισάγει τα απαραίτητα στοιχεία και έαν θέλει την περιγραφή και τους συντελεστές. Εφόσον επιλέξει "Υποβολή Παράστασης" αυτή θα καταχωρηθεί στο σύστημα και θα ανακατευθυνθει πίσω στον πίνακα ελέγχου. Εάν επιλέξει ακύρωση, θα επιστρέψει πίσω στο πίνακα ελέγχου χωρίς καμία αλλαγή.


\subsubsection{Back-end}

\end{document}